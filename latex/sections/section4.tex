Comprehensive evaluation frameworks play a pivotal role in assessing the efficacy of embedding models within RAG systems. These frameworks enable systematic comparison across tasks, fostering deeper insights into model performance while identifying shortcomings and areas for improvement. A variety of benchmarks have been developed to evaluate text embeddings, multilingual capabilities, long-context retrieval, and multimodal tasks, emphasizing versatility and robustness.

The Massive Text Embedding Benchmark (MTEB)\cite{MTEB} is one of the most extensive benchmarking frameworks for text embeddings. Spanning eight tasks and 58 datasets across 112 languages, it evaluates embeddings on tasks such as retrieval, classification, and clustering. While MTEB primarily focuses on cross-task generalizability, frameworks like Benchmarking-IR (BEIR)\cite{BEIR} concentrate on zero-shot information retrieval (IR). BEIR encompasses 18 datasets drawn from heterogeneous text retrieval tasks, offering insights into embedding models’ generalization across out-of-distribution scenarios. Both MTEB and BEIR highlight the ongoing challenge of developing uniformly high-performing text embeddings that balance generalizability and computational efficiency.

Multilingual capabilities in embedding models are another critical evaluation axis, with benchmarks like MKQA\cite{MKQA}, MLDR\cite{bge-m3}, and Tatoeba \cite{Tatoeba} offering diverse perspectives. MKQA is designed for open-domain question answering, featuring 10,000 curated queries with answers aligned across 26 languages, including many low-resource ones. MLDR evaluates embeddings specifically for multilingual long-document retrieval using datasets derived from Wikipedia, mC4, and Wudao. Beyond this, the Tatoeba benchmark enables testing embeddings for linguistic similarity tasks across a broad spectrum of languages, further advancing multilingual evaluation. Beyond these benchmarks, MTEB also applicable for evaluating models on multilingual tasks across its 112 languages.

For long-context retrieval, datasets such as MLDR and NarrativeQA\cite{NarrativeQA} are highly applicable. MLDR assesses the retrieval performance of embedding models over extended text sequences using multilingual long documents, offering high relevance for applications like legal and scientific texts. NarrativeQA, on the other hand, focuses on English documents by providing Wikipedia summaries, links to full-length narratives, and question-answer pairs that challenge models to comprehend longer sequences. Together, these benchmarks address the growing need for effective embeddings in scenarios requiring substantial context length, tightly aligned with the increasing prevalence of such demands in real-world tasks.

In multimodal evaluation, the Massive Multimodal Embedding Benchmark (MMEB)\cite{VLM2Vec} provides a unified framework to assess models across a variety of modalities and tasks. Comprising 36 datasets, it spans classification, information retrieval, visual question answering, and visual grounding, offering scenarios that incorporate text, image, or both as input and output modalities. MMEB presents a realistic testbed for measuring embedding models’ capacity to generalize across modalities, reflecting their applicability to diverse multimodal RAG systems.

These benchmarks collectively cover a broad range of capabilities required by embedding models, including multilingual processing, long-context retrieval, and multimodal understanding. By enabling rigorous and specialized evaluation, they provide the foundation for driving advancements in the design and optimization of embedding models in RAG systems.